\documentclass{beamer}

\mode<presentation> {
	\usetheme{CambridgeUS}
	\usecolortheme{crane}
	\usefonttheme{default}
}

\usepackage{graphicx}
\usepackage{booktabs}
\usepackage{ragged2e}
\usepackage{minted}
\usepackage{lipsum}
\usepackage[export]{adjustbox}

%----------------------------------------------------------------------------------------
%	My Customized Settings
%----------------------------------------------------------------------------------------

\definecolor{UniBlue}{RGB}{83,121,170}
\definecolor{Black}{RGB}{125, 125, 125}
\definecolor{DarkBlue}{RGB}{50, 50, 153}
\definecolor{DarkGray}{RGB}{90,90,90}
\definecolor{LightGray}{RGB}{150,150,150}
\definecolor{TextGreen}{RGB}{115,155,15}
\definecolor{TextOrange}{RGB}{229, 148, 0}
\definecolor{Ocean}{RGB}{23,142,189}
\definecolor{BG}{RGB}{215,215,215}


\setbeamercolor{normal}{fg=DarkGray}
\setbeamercolor{title}{fg=UniBlue}
\setbeamercolor{frametitle}{fg=UniBlue}
\setbeamercolor{structure}{fg=UniBlue}
\setbeamercolor{normal text}{fg=DarkGray,bg=white}
\setbeamercolor{section number projected}{bg=UniBlue,fg=white}

\setbeamertemplate{itemize item}{\scriptsize\raise1.25pt\hbox{\donotcoloroutermaths$\blacktriangleright$}}
\setbeamertemplate{itemize subitem}{\tiny\raise1.5pt\hbox{\donotcoloroutermaths$\bullet$}}
\setbeamertemplate{itemize subsubitem}{\tiny\raise1.5pt\hbox{\donotcoloroutermaths$\blacksqaure$}}
\setbeamercolor{itemize item}{fg=darkred}
\setbeamercolor{itemize subitem}{fg=TextGreen}
\setbeamercolor{itemize subbody}{fg=LightGray}

\setbeamertemplate{enumerate subitem}{\insertenumlabel.\insertsubenumlabel}
\setbeamertemplate{enumerate subsubitem}{\insertenumlabel.\insertsubenumlabel.\insertsubsubenumlabel}
\setbeamertemplate{enumerate mini template}{\insertenumlabel}

\setbeamertemplate{navigation symbols}{}

\newcommand\VeryLargeFont{\fontsize{30}{15}\selectfont}
\newcommand\LargeFont{\fontsize{15}{15}\selectfont}
\newcommand\TinyFont{\fontsize{6}{6}\selectfont}

\setbeamertemplate{frametitle} {
  \nointerlineskip
  \begin{beamercolorbox}[sep=0.15cm,ht=1.3em,wd=\paperwidth]{frametitle}
    \vbox{}\vskip-2ex
    \strut\insertframetitle\strut
    \vskip-0.8ex
  \end{beamercolorbox}
}

\defbeamertemplate*{title page}{customized}[1][] {
	\centering
	\bigskip
	\bigskip
	\bigskip
	\usebeamercolor[fg]{title}\insertsubtitle\par
	\usebeamerfont{title}\inserttitle\par
	\usebeamerfont{subtitle}
	\bigskip
	\usebeamercolor[fg]{normal}
	\usebeamerfont{author}\insertauthor\par
	\usebeamerfont{institute}\insertinstitute\par
	\usebeamerfont{date}\insertdate\par
	\bigskip
	\bigskip
	\bigskip
	\bigskip
	\bigskip
	\usebeamercolor[fg]{titlegraphic}\inserttitlegraphic
}

%----------------------------------------------------------------------------------------
%	TITLE PAGE
%----------------------------------------------------------------------------------------
\title[Operating Systems of the IoT]{An Introduction to IoT Operating Systems}
\author{IoT Lab @ AUT: OS Team}
\institute[] {
  Amirkabir University of Technology \\
  \medskip
  {\small\tt elahejalalpoor@gmail.com}\\
  {\small\tt parham.alvani@gmail.com}
  \medskip
}
\date{\today}
\titlegraphic{\hspace*{5cm}\includegraphics[width=2cm]{figs/aut_logo.jpeg}}

\begin{document}

\begin{frame}
\titlepage
\end{frame}

%----------------------------------------------------------------------------------------
%	PRESENTATION SLIDES
%----------------------------------------------------------------------------------------

%------------------------------------------------
\begin{frame}
	\frametitle{Outline}
	\begin{columns}[c]
		\begin{column}{30cm}
			\vspace{.1cm}
			\begin{itemize}
				\justifying
				\item Part I: IoT OS
				\begin{itemize}
					\item<2-> Introduction
					\item<2-> IoT Requirements \& Challenges
					\item<2-> IoT OS
					\item<2-> Existing OSs				
				\end{itemize}
				\item Part II: IoT Protocol Stack
				\begin{itemize}
					\item<3-> Traditional Stack
					\item<3-> IoT Requirements
					\item<3-> IoT Stack
					\item<3-> Comparison
				\end{itemize}
				\item Part III: IoT Development
				\begin{itemize}
					\item<4-> IoT Lab test
					\item<4-> RIOT environment
					\item<4-> Compilers
					\item<4-> Development environment
				\end{itemize}
				\item Conclusion
			\end{itemize}
		\end{column}
	\end{columns}
\end{frame}

%------------------------------------------------
\begin{frame}
	\frametitle{Outline}
	\begin{columns}[c]
		\begin{column}{30cm}
			\vspace{.1cm}
			\begin{itemize}
				\justifying
				\item Part I: IoT OS
				\begin{itemize}
					\item Introduction
					\item \textcolor{LightGray}{IoT Requirements \& Challenges}
					\item \textcolor{LightGray}{IoT OS}
					\item \textcolor{LightGray}{Existing OSs}
				\end{itemize}
				\item \textcolor{LightGray}{Part II: IoT Protocol Stack}
				\item \textcolor{LightGray}{Part III: IoT Development}
				\item \textcolor{LightGray}{Conclusion}
			\end{itemize}
		\end{column}
	\end{columns}
\end{frame}

%------------------------------------------------
\begin{frame}
	\frametitle{IoT?!!}
	\begin{columns}
		\begin{column}{12cm}
			\begin{block}{\centering\textcolor{darkred}{What is IoT\ldots}}
				\justifying
				[Wikipedia]: The network of physical objects or ``things" embedded with electronics, 	
				software, sensors, and connectivity to enable objects to exchange data with the
				manufacturer, operator and/or other connected devices based on the infrastructure of
				ITU's Global Standards Initiative
			\end{block}

			\begin{block}{\centering\textcolor{darkred}{What is IoT\ldots}}
				\justifying
				[ITU]: A global infrastructure for the information society, enabling advanced services
				by interconnecting (physical and virtual) things based on existing and evolving 		
				interoperable information and communication technologies
			\end{block}
			
			\begin{block}{\centering\textcolor{darkred}{What is IoT\ldots}}
				\justifying
				[WhatIs]: A scenario in which objects, animals or people are provided with unique
				identifiers and the ability to transfer data over a network without requiring 
				human-to-human or human-to-computer interaction.
			\end{block}
			
		\end{column}
	\end{columns}
\end{frame}

%------------------------------------------------
\begin{frame}
	\frametitle{What is the IoT?}
	\begin{itemize}
		\justifying
		\item A \textcolor{TextGreen}{thing} in IoT can be any natural or man-made object can be assigned \textcolor{TextGreen}{IP(v6) address}.
		\item So far, the Internet of Things has been most closely associated with machine-to-machine (M2M) communication.
		\item Although the concept wasn't named until 1999, the Internet of Things has been in development for decades.
		\end{itemize}
	\vspace{.5cm}
	\hspace*{7cm}\includegraphics[width=4cm]{figs/Internet-of-Things-4.jpg}
\end{frame}

%------------------------------------------------
\begin{frame}
	\frametitle{IoT's Applications}
	\begin{itemize}
		\item Environmental monitoring
		\item Infrastructure management
		\item Manufacturing
		\item Energy management
		\item Medical and healthcare systems
		\item Building and home automation
		\item Transportation
		\item \ldots
	\end{itemize}
\end{frame}

%------------------------------------------------
\begin{frame}
	\frametitle{Outline}
	\begin{columns}[c]
		\begin{column}{30cm}
			\vspace{.1cm}
			\begin{itemize}
				\justifying
				\item Part I: IoT OS
				\begin{itemize}
					\item \textcolor{LightGray}{Introduction}
					\item IoT Requirements \& Challenges
					\begin{itemize}
						\item[-] R1: Heterogeneous Hardware Constraints
						\item[-] R2: Autonomy 
						\item[-] R3: Programmability
						\item[-] Effect of the requirements on OS
					\end{itemize}
					\item \textcolor{LightGray}{IoT OS}
					\item \textcolor{LightGray}{Existing OSs}
				\end{itemize}
				\item \textcolor{LightGray}{Part II: IoT Protocol Stack}
				\item \textcolor{LightGray}{Part III: IoT Development}
				\item \textcolor{LightGray}{Conclusion}
			\end{itemize}
		\end{column}
	\end{columns}
\end{frame}

%------------------------------------------------
\begin{frame}
	\frametitle{R1: Heterogeneous Hardware Constraints}
	\begin{itemize}
		\justifying
		\item Memory Requirements
		\item CPU Requirements
		\item Limited Features
		\item Platform Support
	\end{itemize}
	\hspace*{7cm} \includegraphics[width=3.5cm]{figs/Internet-of-Things-3.jpg}
\end{frame}

%------------------------------------------------
\begin{frame}
	\frametitle{Memory Requirements}
	\begin{itemize}
		\justifying
		\item Many of typical IoT devices have very little memory (typically between 5kB and some hundreds of megabytes)
		\item This concerns RAM as well as persistent program storage.
	\end{itemize}

	\begin{columns}
	\column{.8\textwidth}
	\begin{block}{\centering\textcolor{darkred}{Effects on OS}}
		\justifying
		\begin{itemize}
			\item Kernel image should be very small
			\item The RAM footprint should be very low
			\item The OS should be modular!
		\end{itemize}
	\end{block}
	\end{columns}

	\vspace{0.5cm}
	\hspace*{7cm} \includegraphics[width=3.5cm]{figs/RAM.png}
\end{frame}

%------------------------------------------------
\begin{frame}
	\frametitle{CPU Requirements}
	\begin{itemize}
		\justifying
		\item Some of the IoT systems are MCU based (instead of CPU)
		\item Some of the MCUs/CPUs in a IoT system will work at a very low clock cycle.
	\end{itemize}

	\begin{columns}
	\column{.8\textwidth}
	\begin{block}{\centering\textcolor{darkred}{Effects on OS}}
		\justifying
		\begin{itemize}
			\item The complexity of OS must be kept very low
			\item Should be scalable, to accommodate a wide range of different classes of devices
		\end{itemize}
	\end{block}
	\end{columns}
	
	\vspace{0.5cm}
	\hspace*{7cm} \includegraphics[width=3.5cm]{figs/CPU.jpg}
\end{frame}

%------------------------------------------------
\begin{frame}
	\frametitle{Limited Features}
	\begin{itemize}
		\justifying
		\item IoT's hardware may have not advanced components like a Memory Management Unit (MMU) or a Floating-Point Unit (FPU).
	\end{itemize}

	\begin{columns}
	\column{.8\textwidth}
	\begin{block}{\centering\textcolor{darkred}{Effects on OS}}
		\justifying
		\begin{itemize}
			\item Software for IoT must be able to run on constrained HW
			\item Should be scalable, to accommodate a wide range of different classes of devices
		\end{itemize}
	\end{block}
	\end{columns}
		
	\vspace{0.5cm}
	\hspace*{7cm} \includegraphics[width=3.5cm]{figs/Features.jpg}
\end{frame}

%------------------------------------------------
\begin{frame}
	\frametitle{Platform Support}
	\begin{itemize}
		\justifying
		\item IoT platforms may have very limitted resources; e.g., battery, IO, storage, ...
		\item IoT platforms may composed of widly different components
	\end{itemize}

	\begin{columns}
	\column{.8\textwidth}
	\begin{block}{\centering\textcolor{darkred}{Effects on OS}}
		\justifying
		\begin{itemize}
			\item Must be able to leverage the capabilities of less constrained platforms
			\item Should be scalable, to accommodate a wide range of different classes of devices
		\end{itemize}
	\end{block}
	\end{columns}
	\vspace{0.5cm}
	\hspace*{7cm} \includegraphics[width=3.5cm]{figs/hw-platform.jpg}
\end{frame}

%------------------------------------------------
\begin{frame}
	\frametitle{R2: Autonomy}
	\begin{columns}[c]
		\begin{column}{30cm}
			\vspace{.1cm}
			\begin{itemize}
				\justifying
				\item Energy efficiency
				\item Adaptive Network Stack
				\item Reliability
			\end{itemize}
		\end{column}
	\end{columns}
	\hspace*{7cm} \includegraphics[width=3.5cm]{figs/Internet-of-Things-3.jpg}
\end{frame}

%------------------------------------------------
\begin{frame}
	\frametitle{Energy efficiency}
	\begin{columns}[c]
		\begin{column}{30cm}
			\vspace{.1cm}
			\begin{itemize}
				\justifying
				\item The software must exploit the power saving features\\
				of the hardware and allow for large sleep
				cycles as much as possible.
			\end{itemize}
		\end{column}
	\end{columns}
	\vspace{1cm}
	\hspace*{7cm} \includegraphics[width=3.5cm]{figs/Energy-efficiency.jpg}
\end{frame}

%------------------------------------------------
\begin{frame}
	\frametitle{Adaptive Network Stack}
	\begin{columns}[c]
		\begin{column}{30cm}
			\vspace{.1cm}
			\begin{itemize}
				\justifying
				\item The network stack should provide full-fledged TCP/IP\\
				implementations as well as a 6LoWPAN stack aiming\\
				for more constrained devices.
				\item It should also be modular in a way that the protocols\\
				at each layer can be easily replaced. 
			\end{itemize}
		\end{column}
	\end{columns}
	\vspace{1cm}
	\hspace*{3cm} \includegraphics[width=9cm]{figs/tcp-ip-stack.png}
\end{frame}

%------------------------------------------------
\begin{frame}
	\frametitle{Reliability}
	\begin{columns}[c]
		\begin{column}{30cm}
			\vspace{.1cm}
			\begin{itemize}
				\justifying
				\item IoT systems are often deployed in critical applications\\
				in which physical access is difficult and related to high costs\\
				in many cases. For that reason, it is important that the system\\
				is robust and thus that the operating system runs very reliably.
			\end{itemize}
		\end{column}
	\end{columns}
	\vspace{1cm}
	\hspace*{7cm} \includegraphics[width=3.5cm]{figs/reliability.jpg}
\end{frame}

%------------------------------------------------
\begin{frame}
	\frametitle{R3: Programmability}
	\begin{columns}[c]
		\begin{column}{30cm}
			\vspace{.1cm}
			\begin{itemize}
				\justifying
				\item Standard API
				\item Standard Programming Languages
			\end{itemize}
		\end{column}
	\end{columns}
	\hspace*{7cm} \includegraphics[width=3.5cm]{figs/Internet-of-Things-3.jpg}
\end{frame}

%------------------------------------------------
\begin{frame}
	\frametitle{Standard API}
	\begin{columns}[c]
		\begin{column}{30cm}
			\vspace{.1cm}
			\begin{itemize}
				\justifying
				\item In order to ease software development and simplify\\
				the porting of existing software, a standard programming interface\\
				such as POSIX or STL should be provided.
			\end{itemize}
		\end{column}
	\end{columns}
	\vspace{1cm}
	\hspace*{7cm} \includegraphics[width=3.5cm]{figs/ieee-posix-certified.jpg}
\end{frame}

%------------------------------------------------
\begin{frame}
	\frametitle{Standard Programming Languages}
	\begin{columns}[c]
		\begin{column}{30cm}
			\vspace{.1cm}
			\begin{itemize}
				\justifying
				\item Support for standard high level programming languages,\\
				such as C++, is vital.
			\end{itemize}
		\end{column}
	\end{columns}
	\vspace{1cm}
	\hspace*{7cm} \includegraphics[width=3.5cm]{figs/C.png}
\end{frame}

%------------------------------------------------
\begin{frame}
	\frametitle{IoT Challenges}
	\colorbox{green}{\textcolor{red}{References are useful \& needed}}	
	\begin{columns}[c]
		\begin{column}{30cm}
			\vspace{.1cm}
			\begin{itemize}
				\justifying
				\item Heterogeneous hardware
				\item Slow CPU, often no FPU
				\item Little memory, often no MMU
				\item Limited energy resources
				\item Robustness and self-organization
				\item Real-Time requirements
			\end{itemize}
		\end{column}
	\end{columns}
	\hspace*{7cm} \includegraphics[width=3.5cm]{figs/Internet-of-Things-3.jpg}
\end{frame}

%------------------------------------------------
\begin{frame}
	\frametitle{Effect of the requirements on OS}
	\begin{columns}[c]
		\begin{column}{30cm}
			\vspace{.1cm}
			\begin{itemize}
				\justifying
				\item \textcolor{TextOrange}{Scalable}, to accommodate a wide range of different
				 classes of devices
				\item \textcolor{TextGreen}{Modular}, so you can choose only the components you\\
				need to meet tight RAM requirements
				\item \textcolor{TextOrange}{Connected}, so you can move data in and out of the\\
				device via Wi-Fi, Ethernet, USB, or Bluetooth.
				\item \textcolor{TextGreen}{Reliable}, so your device can be certified for safety-critical
				applications
			\end{itemize}
		\end{column}
	\end{columns}
\end{frame}

%------------------------------------------------
\begin{frame}
	\frametitle{Outline}
	\begin{columns}[c]
		\begin{column}{30cm}
			\vspace{.1cm}
			\begin{itemize}
				\justifying
				\item Part I: IoT OS
				\begin{itemize}
					\item \textcolor{LightGray}{Introduction}
					\item \textcolor{LightGray}{IoT Requirements \& Challenges}
					\item {IoT OS}
					\begin{itemize}
						\item[-] General OS vs IoT OS
						\item[-] What are the main requirements in IoT OS
						\item[-] What are the main components in IoT OS
				\end{itemize}
					\item \textcolor{LightGray}{Existing OSs}
				\end{itemize}
				\item \textcolor{LightGray}{Part II: IoT Protocol Stack}
				\item \textcolor{LightGray}{Part III: IoT Development}
				\item \textcolor{LightGray}{Conclusion}
			\end{itemize}
		\end{column}
	\end{columns}
\end{frame}

%------------------------------------------------
\begin{frame}
	\frametitle{General OS}
	\vspace{.5cm}
	\hspace*{1.5cm} \includegraphics[width=10cm]{figs/os-components.jpg}
\end{frame}

%------------------------------------------------
\begin{frame}
	\frametitle{IoT OS}
	\vspace{.5cm}
	\hspace*{1cm} \includegraphics[width=10cm]{figs/mbed-components.png}
\end{frame}

%------------------------------------------------
\begin{frame}
	\frametitle{Multi-Tasking, Thread Model (IoT OS)}
	\begin{columns}[c]
		\begin{column}{30cm}
			\vspace{.1cm}
			\begin{itemize}
				\justifying
				\item Most RTOS products on the market are thread model.
				\item \textcolor{TextGreen}{Tasks} are now called \textcolor{TextGreen}{threads}.
				\item All the \textcolor{TextOrange}{tasks} code and data occupy
				\textcolor{Ocean}{the same address space},\\
				along with that of the RTOS itself.
				\item Or every \textcolor{TextOrange}{tasks} can run in its own thread and \\
				\textcolor{Ocean}{has its own memory stack}.
			\end{itemize}
		\end{column}
	\end{columns}
	\vspace{.5cm}
	\hspace*{5.5cm} \includegraphics[width=3.5cm]{figs/thread-model.jpg}
\end{frame}

%------------------------------------------------
\begin{frame}
	\frametitle{What are the main requirements in IoT OS}
	\begin{columns}[c]
		\begin{column}{30cm}
			\vspace{.1cm}
			\begin{itemize}
				\justifying
				\item IoT Protocol Stack Support
				\item Efficient Memory Managing
				\item Real-Time Task Scheduling
			\end{itemize}
		\end{column}
	\end{columns}
\end{frame}

%------------------------------------------------
\begin{frame}
	\frametitle{What are the main components in IoT OS}
	\begin{columns}[c]
		\begin{column}{30cm}
			\vspace{.1cm}
			\begin{itemize}
				\justifying
				\item Networking
				\item Memory Manager
				\item Task Scheduler
			\end{itemize}
		\end{column}
	\end{columns}
\end{frame}

%------------------------------------------------
\begin{frame}
	\frametitle{Outline}
	\begin{columns}[c]
		\begin{column}{30cm}
			\vspace{.1cm}
			\begin{itemize}
				\justifying
				\item Part I: IoT OS
				\begin{itemize}
					\item \textcolor{LightGray}{Introduction}
					\item \textcolor{LightGray}{IoT Requirements \& Challenges}
					\item \textcolor{LightGray}{IoT OS}
					\item {Existing OSs}
					\begin{itemize}
						\item[-] OS Classification
						\item[-] Overview of Open Source OSs
						\item[-] Overview of Closed Source OSs
						\item[-] Why Not Linux?
					\end{itemize}
				\end{itemize}
				\item \textcolor{LightGray}{Part II: IoT Protocol Stack}
				\item \textcolor{LightGray}{Part III: IoT Development}
				\item \textcolor{LightGray}{Conclusion}
			\end{itemize}
		\end{column}
	\end{columns}
\end{frame}

%------------------------------------------------
\begin{frame}
	\frametitle{OS Classification}
	\begin{columns}[c]
		\begin{column}{30cm}
			\vspace{.1cm}
			\begin{itemize}
				\justifying
				\item Design Aspects for an IoT OS
				\begin{itemize}
					\item Monolithic fashion
					\item Layered approach
					\item Microkernel architecture
				\end{itemize}
			\end{itemize}
		\end{column}
	\end{columns}
\end{frame}

%------------------------------------------------
\begin{frame}
	\frametitle{OS Classification}
	\begin{columns}[c]
		\begin{column}{30cm}
			\vspace{.1cm}
			\begin{itemize}
				\justifying
				\item Programming Model for an IoT OS
				\begin{itemize}
					\item All tasks are executed within the same context and have\\
					no segmentation of the memory address space.
					\item Every process can run in its own thread and has its own\\
					memory stack.
				\end{itemize}
			\end{itemize}
		\end{column}
	\end{columns}
\end{frame}

%------------------------------------------------
\begin{frame}
	\frametitle{Overview of Open Source OSs}
	\begin{columns}[c]
		\begin{column}{30cm}
			\vspace{.1cm}
			\begin{itemize}
				\justifying
				\item \textcolor{blue}{\href{http://www.freertos.org}{FreeRTOS}}
				\item \textcolor{blue}{\href{http://www.riot-os.org}{RIOT}}
				\item \textcolor{blue}{\href{http://www.contiki-os.org}{Contiki}}
				\item TinyOS
				\item Embedded Linux
				\item \textcolor{blue}{\href{http://www.openwsn.org}{OpenWSN}}
			\end{itemize}
		\end{column}
	\end{columns}
	\vspace{.5cm}
	\hspace*{5.5cm} \includegraphics[width=5cm]{figs/Internet-of-Things-2.jpg}
\end{frame}

%------------------------------------------------
\begin{frame}
	\frametitle{FreeRTOS}
	\begin{columns}[c]
		\begin{column}{30cm}
			\vspace{.1cm}
			\begin{itemize}
				\justifying
				\item FreeRTOS is designed to be \textcolor{TextOrange}{small}
				and \textcolor{TextOrange}{simple}.
				\item The kernel itself consists of only three or four C files.
				\item It provides methods for multiple threads or tasks, mutexes,\\
				semaphores and software timers.
				\item Key features are \textcolor{TextGreen}{very small memory footprint},
				\textcolor{TextGreen}{low overhead},\\
				and \textcolor{TextGreen}{very fast execution}.
			\end{itemize}
		\end{column}
	\end{columns}
	\vspace{.5cm}
	\hspace*{5.5cm} \includegraphics[width=5cm]{figs/freertos-logo.jpg}	
\end{frame}

%------------------------------------------------
\begin{frame}
	\frametitle{RIOT}
	\begin{columns}[c]
		\begin{column}{30cm}
			\vspace{.1cm}
			\begin{itemize}
				\justifying
				\item RIOT is a \textcolor{TextOrange}{real-time}
				\textcolor{TextOrange}{multi-threading} operating system.
				\item RIOT implements a \textcolor{TextGreen}{microkernel} architecture
				\item RIOT is based on design objectives including:
				\begin{itemize}
					\justifying
					\item Energy-Efficiency
					\item Reliability
					\item Real-Time Capabilities
					\item Small Memory Footprint
					\item Modularity
					\item Uniform API Access\\
					independent of the underlying hardware\\
					(this API offers partial POSIX compliance)
				\end{itemize}
			\end{itemize}
		\end{column}
	\end{columns}
	\vspace{.25cm}
	\hspace*{7cm} \includegraphics[width=5cm]{figs/riot-logo.png}
\end{frame}

%------------------------------------------------
\begin{frame}
	\frametitle{Contiki}
	\begin{columns}[c]
		\begin{column}{30cm}
			\vspace{.1cm}
			\begin{itemize}
				\justifying
				\item Contiki is an open source operating system for \textcolor{TextOrange}{networked},\\
				\textcolor{TextOrange}{memory-constrained} systems
				\item Contiki provides three network mechanisms:
				\begin{itemize}
					\justifying
					\item The uIP stack, which provides IPv4 networking,
					\item The uIPv6 stack, which provides IPv6 networking,
					\item The Rime stack, which is a set of custom lightweight networking protocols\\
					designed specifically for low-power wireless networks.
				\end{itemize}
			\end{itemize}
		\end{column}
	\end{columns}
	\vspace{.5cm}
	\hspace*{5.5cm} \includegraphics[width=5cm]{figs/contiki-logo.png}
\end{frame}

%------------------------------------------------
\begin{frame}
	\frametitle{TinyOS}
	\begin{columns}[c]
		\begin{column}{30cm}
			\vspace{.1cm}
			\begin{itemize}
				\justifying
				\item TinyOS is a \textcolor{TextOrange}{component-based} operating system and platform\\
				targeting wireless sensor networks.
				\item TinyOS is an embedded operating system written in the \textcolor{TextOrange}{nesC}\\
				\textcolor{TextOrange}{programming language} as a set of cooperating tasks and processes.
			\end{itemize}
		\end{column}
	\end{columns}
	\vspace{.5cm}
	\hspace*{5.5cm} \includegraphics[width=5cm]{figs/tinyos-logo.jpg}
\end{frame}

%------------------------------------------------
\begin{frame}
	\frametitle{Embedded Linux}
	\begin{columns}[c]
		\begin{column}{30cm}
			\vspace{.1cm}
			\begin{itemize}
				\justifying
				\item Embedded Linux is created using OpenEmbedded,\\
				the build framework for embedded Linux.
				\item OpenEmbedded offers a best-in-class cross-compile environment.
			\end{itemize}
		\end{column}
	\end{columns}
	\vspace{.5cm}
	\hspace*{5.5cm} \includegraphics[width=5cm]{figs/linux-logo.jpeg}
\end{frame}

%------------------------------------------------
\begin{frame}
	\frametitle{OpenWSN}
	\begin{columns}[c]
		\begin{column}{30cm}
			\vspace{.1cm}
			\begin{itemize}
				\justifying
				\item The goal of the OpenWSN project is to provide open-source\\
				implementations of a complete protocol stack based on Internet of\\
				Things standards, on a variety of software and hardware platforms.
			\end{itemize}
		\end{column}
	\end{columns}
	\vspace{.5cm}
	\hspace*{5.5cm} \includegraphics[width=5cm]{figs/openwsn-logo.png}
\end{frame}

%------------------------------------------------
\begin{frame}
	\frametitle{Comparison}
	\begin{columns}[c]
		\begin{column}{30cm}
			\hspace{0.9cm}
			\begin{tabular}{| c | c | c | c | c |}
				\hline
				OS & Min RAM & Min ROM & C Support & C++ Support \\ \hline
				Contiki & $< 2kB$ & $< 30kB$ & \textcolor{TextOrange}{Partial support} &
				\textcolor{red}{No support} \\ \hline
				Tiny OS & $< 1kB$ & $< 4kB$ & \textcolor{red}{No support} &
				\textcolor{red}{No support} \\ \hline
				Linux & $\sim 1MB$ & $\sim 1MB$ & \textcolor{TextGreen}{Full support} &
				\textcolor{TextGreen}{Full support} \\ \hline
				RIOT & $\sim 1.5kB$ & $\sim 5kB$ & \textcolor{TextGreen}{Full support} &
				\textcolor{TextGreen}{Full support} \\ \hline
			\end{tabular}
		\end{column}
	\end{columns}
	\vspace{.5cm}
	\hspace*{1cm}
	\includegraphics[width=2cm]{figs/contiki-logo.png}
	\hspace*{.5cm}
	\includegraphics[width=2cm]{figs/tinyos-logo.jpg}
	\hspace*{.5cm}
	\includegraphics[width=2cm]{figs/linux-logo.jpeg}
	\hspace*{.5cm}
	\includegraphics[width=2cm]{figs/riot-logo.png}
\end{frame}

%------------------------------------------------
\begin{frame}
	\frametitle{Comparison}
	\begin{columns}[c]
		\begin{column}{30cm}
			\hspace{0.9cm}
			\begin{tabular}{| c | c | c | c |}
				\hline
				OS & Multi-Threading & Modularity & Real-Time \\ \hline
				Contiki & \textcolor{TextOrange}{Partial support} &
				\textcolor{TextOrange}{Partial support} &
				\textcolor{TextOrange}{Partial support} \\ \hline
				Tiny OS & \textcolor{TextOrange}{Partial support} &
				\textcolor{red}{No support} &
				\textcolor{red}{No support} \\ \hline
				Linux & \textcolor{TextGreen}{Full support} &
				\textcolor{TextOrange}{Partial support} &
				\textcolor{TextOrange}{Partial support} \\ \hline
				RIOT & \textcolor{TextGreen}{Full support} &
				\textcolor{TextGreen}{Full support} &
				\textcolor{TextGreen}{Full support} \\ \hline
			\end{tabular}
		\end{column}
	\end{columns}
	\vspace{.5cm}
	\hspace*{1cm}
	\includegraphics[width=2cm]{figs/contiki-logo.png}
	\hspace*{.5cm}
	\includegraphics[width=2cm]{figs/tinyos-logo.jpg}
	\hspace*{.5cm}
	\includegraphics[width=2cm]{figs/linux-logo.jpeg}
	\hspace*{.5cm}
	\includegraphics[width=2cm]{figs/riot-logo.png}
\end{frame}

%------------------------------------------------
\begin{frame}
	\frametitle{Operating Systems Availability}
	\begin{columns}[c]
		\begin{column}{30cm}
			\hspace{0.9cm}
			\begin{tabular}{| c | c | c | c |}
				\hline
				OS & Wsn430 Node & M3 Node & A8 Node \\ \hline
				Contiki & \textcolor{TextGreen}{Full support} &
				\textcolor{TextGreen}{Full support} &
				\textcolor{red}{No support} \\ \hline
				Tiny OS & \textcolor{TextGreen}{Full support} &
				\textcolor{red}{No support} &
				\textcolor{red}{No support} \\ \hline
				Linux & \textcolor{red}{No support} &
				\textcolor{red}{No support} &
				\textcolor{TextGreen}{Full support} \\ \hline
				RIOT & \textcolor{TextGreen}{Full support} &
				\textcolor{TextGreen}{Full support} &
				\textcolor{red}{No support} \\ \hline
			\end{tabular}
		\end{column}
	\end{columns}
	\vspace{.5cm}
	\hspace*{1cm}
	\includegraphics[width=2cm]{figs/contiki-logo.png}
	\hspace*{.5cm}
	\includegraphics[width=2cm]{figs/tinyos-logo.jpg}
	\hspace*{.5cm}
	\includegraphics[width=2cm]{figs/linux-logo.jpeg}
	\hspace*{.5cm}
	\includegraphics[width=2cm]{figs/riot-logo.png}
\end{frame}

%------------------------------------------------
\begin{frame}
	\frametitle{Overview of Closed Source OSs}
	\begin{columns}[c]
		\begin{column}{30cm}
			\vspace{.1cm}
			\begin{itemize}
				\justifying
				\item \textcolor{blue}{\href{https://mbed.org/}{ARM mbed}}
				\item Huawei LiteOS
				\item Google Brillo
			\end{itemize}
		\end{column}
	\end{columns}
	\vspace{.5cm}
	\hspace*{5.5cm} \includegraphics[width=5cm]{figs/Internet-of-Things-2.jpg}
\end{frame}

%------------------------------------------------
\begin{frame}
	\frametitle{ARM mbed}
	\begin{columns}[c]
		\begin{column}{30cm}
			\vspace{.1cm}
			\begin{itemize}
				\justifying
				\item Automation of power management
				\item Software asset protection and secure firmware updates for\\
				device security \& management
				\item Connectivity protocol stack support for Bluetooth® low energy,\\
				Cellular, Ethernet, Wi-fi, Zigbee IP, Zigbee NAN, 6LoWPAN
			\end{itemize}
		\end{column}
	\end{columns}
	\vspace{.5cm}
	\hspace*{5.5cm} \includegraphics[width=5cm]{figs/ARM-mbed-logo.png}
\end{frame}

%------------------------------------------------
\begin{frame}
	\frametitle{Huawei LiteOS}
	\begin{columns}[c]
		\begin{column}{30cm}
			\vspace{.1cm}
			\begin{itemize}
				\justifying
				\item The company says that its \textcolor{TextOrange}{LiteOS} is
				the \textcolor{TextGreen}{lightest} software of\\
				its kind and can be used to power a range of smart devices
			\end{itemize}
		\end{column}
	\end{columns}
	\vspace{.5cm}
	\hspace*{5.5cm} \includegraphics[width=5cm]{figs/huawei-liteos-logo.jpg}
\end{frame}

%------------------------------------------------
\begin{frame}
	\frametitle{Google Brillo}
	\begin{columns}[c]
		\begin{column}{30cm}
			\vspace{.1cm}
			\begin{itemize}
				\justifying
				\item Brillo is \textcolor{TextGreen}{derived} from Android
				but \textcolor{TextGreen}{polished} to just the lower levels.
				\item It supports Wi-Fi, Bluetooth Low Energy, and other Android things.
			\end{itemize}
		\end{column}
	\end{columns}
	\vspace{.5cm}
	\hspace*{5.5cm} \includegraphics[width=5cm]{figs/google-brillo-logo.jpeg}
\end{frame}

%------------------------------------------------
\begin{frame}
	\frametitle{Why Not Linux?}
	\begin{columns}[c]
		\begin{column}{12cm}
			\vspace{1cm}
			\begin{block}{\centering\textcolor{darkred}{Real-Time Linux}}
				\justifying
				Controlling a laser with Linux is crazy, but everyone in this room is crazy
				in his own way. So if you want to use Linux to control an industrial welding
				laser, I have no problem with your using PREEMPT\_RT.
				\vspace{.2cm}
				\hspace*{9.5cm}\footnotesize{- Linus Torvalds}
			\end{block}
		\end{column}
	\end{columns}
	\vspace{.5cm}
	\hspace*{.5cm}
	\includegraphics[width=2cm]{figs/preempt-rt.png}
	\hspace*{3cm}
	\includegraphics[width=5cm]{figs/linus-torvalds.jpg}
\end{frame}

%------------------------------------------------
\begin{frame}
	\frametitle{Why Not Linux?}
	\begin{columns}[c]
		\begin{column}{30cm}
			\vspace{.1cm}
			\begin{itemize}
				\justifying
				\item Linux certainly is a robust, developer-friendly OS
				\item Linux has a disadvantage when compared to a real-time operating system:
				\begin{itemize}
					\justifying
					\item Memory footprint
					\item It simply will not run on 8 or 16-bit MCUs
				\end{itemize}
				\item Linux will certainly have many uses in embedded devices, particularly\\
				ones that provide graphically rich user interfaces.
				\item There are thousands of applications for which Linux is ill suited.
			\end{itemize}
		\end{column}
	\end{columns}
	\vspace{.5cm}
	\hspace*{5.5cm} \includegraphics[width=5cm]{figs/tux-sad.png}
\end{frame}

%------------------------------------------------
\begin{frame}
	\frametitle{Outline}
	\begin{columns}[c]
		\begin{column}{30cm}
			\vspace{.1cm}
			\begin{itemize}
				\justifying
				\item \textcolor{LightGray}{Part I: IoT OS}
				\item Part II: IoT Protocol Stack
				\begin{itemize}
					\item Traditional Stack
					\item IoT Requirements
					\item IoT Stack
					\item Comparison
				\end{itemize}
				\item \textcolor{LightGray}{Part III: IoT Development}
				\item \textcolor{LightGray}{Conclusion}
			\end{itemize}
		\end{column}
	\end{columns}
\end{frame}

%------------------------------------------------
\begin{frame}
	\frametitle{Protocol stack}
	\begin{columns}[c]
		\begin{column}{30cm}
			\vspace{.1cm}
			\begin{itemize}
				\justifying
				\item Can you build an IoT system with familiar Web technologies?
			\end{itemize}
		\end{column}
	\end{columns}
\end{frame}

%------------------------------------------------
\begin{frame}
	\frametitle{Protocol stack}
	\begin{columns}[c]
		\begin{column}{30cm}
			\vspace{.1cm}
			\begin{itemize}
				\justifying
				\item Can you build an IoT system with familiar Web technologies?
				\item Yes you can, although the result would not be as \textcolor{Ocean}{efficient} as\\
				with the \textcolor{TextGreen}{newer protocols}.
			\end{itemize}
		\end{column}
	\end{columns}
\end{frame}

%------------------------------------------------
\begin{frame}
	\frametitle{Traditional Stack}
	\begin{columns}[c]
		\begin{column}{30cm}
			\vspace{.1cm}
			\begin{itemize}
				\justifying
				\item Existing Internet protocols such as HTTP and TCP are not optimized\\
				for very \textcolor{TextOrange}{low-power communication}.
				\item Energy is wasted by transmission of \textcolor{TextGreen}{unneeded data},
				\textcolor{TextGreen}{protocol overhead},\\
				and \textcolor{TextGreen}{non-optimized communication patterns}.
			\end{itemize}
		\end{column}
	\end{columns}
\end{frame}

%------------------------------------------------
\begin{frame}
	\frametitle{IoT Requirements}
	\begin{columns}[c]
		\begin{column}{30cm}
			\vspace{.1cm}
			\begin{itemize}
				\justifying
				\item A Low Power Communication Stack.
				\item A Highly Reliable Communication Stack.
				\item An Internet-Enabled Communication Stack.
			\end{itemize}
		\end{column}
	\end{columns}
\end{frame}

%------------------------------------------------
\begin{frame}
	\frametitle{IoT Stack}
	\begin{columns}[c]
		\begin{column}{30cm}
			\vspace{.1cm}
			\begin{itemize}
				\justifying
				\item \textsc{Low-Power Physical Layer} − \textcolor{TextOrange}{IEEE 802.15.4}
				\item \textsc{Power-Saving Link Layer} − \textcolor{TextOrange}{IEEE 802.15.4E}
				\item \textsc{Connecting To The Internet} - \textcolor{TextGreen}{IETF 6LoWPAN}
				\item \textsc{Routing} - \textcolor{TextGreen}{IETF ROLL}
				\item \textsc{Transport Layer and Above} - \textcolor{TextGreen}{IETF CoAP}
			\end{itemize}
		\end{column}
	\end{columns}
\end{frame}

%------------------------------------------------
\begin{frame}
	\frametitle{IoT Stack}
	\vspace{.25cm}
	\hspace*{.75cm} \includegraphics[width=10.5cm]{figs/iot-network-protocols.jpeg}
\end{frame}

%------------------------------------------------
\begin{frame}
	\frametitle{Comparison}
	\vspace{.5cm}
	\hspace*{1.5cm} \includegraphics[width=10cm]{figs/Web-and-IoT-Stacks-1.png}
\end{frame}

%------------------------------------------------
\begin{frame}
	\frametitle{Comparison}
	\vspace{.5cm}
	\hspace*{.5cm} \includegraphics[width=12cm]{figs/Web-and-IoT-Stacks-2.png}
\end{frame}
%------------------------------------------------
\begin{frame}
	\frametitle{Outline}
	\begin{columns}[c]
		\begin{column}{30cm}
			\vspace{.1cm}
			\begin{itemize}
				\justifying
				\item \textcolor{LightGray}{Part I: IoT OS}
				\item \textcolor{LightGray}{Part II: IoT Protocol Stack}
				\item Part III: IoT Development
				\begin{itemize}
					\item IoT Lab test
					\item RIOT environment
					\item Compilers
					\item Development environment
				\end{itemize}
				\item \textcolor{LightGray}{Conclusion}
			\end{itemize}
		\end{column}
	\end{columns}
\end{frame}

%------------------------------------------------
\begin{frame}
	\frametitle{IoT Lab test}
	\begin{columns}[c]
		\begin{column}{30cm}
			\vspace{.1cm}
			\begin{itemize}
				\justifying
				\item A scientific testbed
				\item Different topologies and environments
				\item Different nodes
				\item A part of FIT
			\end{itemize}
		\end{column}
	\end{columns}
	\vspace{.5cm}
	\hspace*{5.5cm} \includegraphics[width=5cm]{figs/fit-iot-lab.jpg}
\end{frame}

%------------------------------------------------
\begin{frame}
	\frametitle{A scientific testbed}
	\begin{columns}[c]
		\begin{column}{30cm}
			\vspace{.1cm}
			\begin{itemize}
				\justifying
				\item IoT-LAB provides full control of network nodes and direct access\\
				to the gateways to which nodes are connected, allowing researchers\\
				to monitor nodes energy consumption and network-related metrics.
			\end{itemize}
		\end{column}
	\end{columns}
	\vspace{.5cm}
	\hspace*{5.5cm} \includegraphics[width=5cm]{figs/iot-lab-1.png}
\end{frame}

%------------------------------------------------
\begin{frame}
	\frametitle{Different topologies and environments}
	\begin{columns}[c]
		\begin{column}{30cm}
			\vspace{.1cm}
			\begin{itemize}
				\justifying
				\item IoT-LAB testbeds are located at six different sites across France which\\
				gives forward access to \textcolor{TextOrange}{2728 wireless sensors nodes}.
			\end{itemize}
		\end{column}
	\end{columns}
	\vspace{.5cm}
	\hspace*{5.5cm} \includegraphics[width=5cm]{figs/iot-lab-2.png}
\end{frame}

%------------------------------------------------
\begin{frame}
	\frametitle{Different nodes}
	\begin{columns}[c]
		\begin{column}{30cm}
			\vspace{.1cm}
			\begin{itemize}
				\justifying
				\item The IoT-LAB hardware infrastructure consists of a set of IoT-LAB nodes.
				\item A global networking backbone provides power and connectivity to\\
				all IoT-LAB nodes and guaranties the out of band signal network\\
				needed for command purposes and monitoring feedback.
			\end{itemize}
		\end{column}
	\end{columns}
	\vspace{.5cm}
	\hspace*{5.5cm} \includegraphics[width=5cm]{figs/iot-lab-3.png}
\end{frame}

%------------------------------------------------
\begin{frame}
	\frametitle{A part of FIT}
	\begin{columns}[c]
		\begin{column}{30cm}
			\vspace{.1cm}
			\begin{itemize}
				\justifying
				\item IoT-LAB is a part of the FIT (Future Internet of the Things) platform.
				\item FIT is a set of complementary components that enable experimentation\\
				on innovative services for academic and industrial users.
			\end{itemize}
		\end{column}
	\end{columns}
	\vspace{.5cm}
	\hspace*{5.5cm} \includegraphics[width=5cm]{figs/iot-lab-4.png}
\end{frame}

%------------------------------------------------
\begin{frame}
	\frametitle{RIOT environment}
	\begin{columns}[c]
		\begin{column}{30cm}
			\vspace{.1cm}
			\begin{itemize}
				\justifying
				\item RIOT features the native port with networking support.
				\item This allows you to run any RIOT application on your Linux or Mac\\
				computer and setup a virtual connection between these processes.
			\end{itemize}
		\end{column}
	\end{columns}
	\vspace{.5cm}
	\hspace*{5.5cm} \includegraphics[width=5cm]{figs/riot-logo.png}
\end{frame}

%------------------------------------------------
\begin{frame}
	\frametitle{Compilers}
	\begin{columns}[c]
		\begin{column}{30cm}
			\vspace{.1cm}
			\begin{itemize}
				\justifying
				\item Family: ARM
				\begin{itemize}
					\item gcc-arm-embedded toolchain
					\item CodeBench toolchain
					\item Linaro toolchain
				\end{itemize}
				\item Family: ATmega
				\begin{itemize}
					\item Atmel AVR Toolchain
				\end{itemize}
				\item Family: MSP430
				\begin{itemize}
					\item MSPGCC toolchain
				\end{itemize}
			\end{itemize}
		\end{column}
	\end{columns}
	\vspace{.5cm}
	\hspace*{5.5cm} \includegraphics[width=5cm]{figs/parse-tree.png}
\end{frame}

%------------------------------------------------
\begin{frame}
	\frametitle{Development environment}
	\begin{columns}[c]
		\begin{column}{30cm}
			\vspace{.1cm}
			\begin{itemize}
				\justifying
				\item Most of the IoT OS developed on \textcolor{TextOrange}{Linux} and\\
				use \textcolor{TextGreen}{traditional make} as build system.
			\end{itemize}
		\end{column}
	\end{columns}
	\vspace{.5cm}
	\hspace*{.5cm}
	\includegraphics[width=5cm]{figs/linux-logo.jpeg}
	\includegraphics[width=5cm]{figs/gun-logo.png}
\end{frame}

%------------------------------------------------
\begin{frame}
	\frametitle{Outline}
	\begin{columns}[c]
		\begin{column}{30cm}
			\vspace{.1cm}
			\begin{itemize}
				\justifying
				\item \textcolor{LightGray}{Part I: IoT OS}
				\item \textcolor{LightGray}{Part II: IoT Protocol Stack}
				\item \textcolor{LightGray}{Part III: IoT Development}
				\item Conclusion
				\begin{itemize}
					\item Open problems
					\item Event-Driven, Non-Blocking I/O Model
				\end{itemize}				
			\end{itemize}
		\end{column}
	\end{columns}
\end{frame}

%------------------------------------------------
\begin{frame}
	\frametitle{Open problems}
	\begin{columns}[c]
		\begin{column}{30cm}
			\vspace{.1cm}
			\begin{itemize}
				\justifying
				\item Ideally, the capabilities of a full-fledged OS should be available\\
				on all IoT devices.
				\item Native Multi-Threading
				\item Hardware Abstraction
				\item Dynamic Memory Management
				\item Fulfill Strict Energy Efficiency
			\end{itemize}
		\end{column}
	\end{columns}
	\vspace{.5cm}
	\hspace*{5.5cm} \includegraphics[width=5cm]{figs/open-problems.jpg}
\end{frame}

%------------------------------------------------
\begin{frame}
	\frametitle{Event-Driven, Non-Blocking I/O Model}
	\begin{columns}[c]
		\begin{column}{30cm}
			\vspace{.1cm}
			\begin{itemize}
				\justifying
				\item Networking Event-Driven
				\item Non-Blocking I/O
			\end{itemize}
		\end{column}
	\end{columns}
	\vspace{.5cm}
	\hspace*{5.5cm} \includegraphics[width=5cm]{figs/io.png}
\end{frame}

%------------------------------------------------
\begin{frame}
	\vspace{1cm}
	\begin{Huge}
		\begin{center}
			\usebeamercolor[fg]{title}Questions?
		\end{center}
	\end{Huge}
\end{frame}

\end{document}
