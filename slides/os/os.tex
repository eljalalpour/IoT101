\documentclass{beamer}

\mode<presentation> {
	\usetheme{CambridgeUS}
	\usecolortheme{crane}
	\usefonttheme{default}
}

\usepackage{graphicx}
\usepackage{booktabs}
\usepackage{ragged2e}
\usepackage{minted}
\usepackage{lipsum}
\usepackage[export]{adjustbox}

%----------------------------------------------------------------------------------------
%	My Customized Settings
%----------------------------------------------------------------------------------------

\definecolor{UniBlue}{RGB}{83,121,170}
\definecolor{Black}{RGB}{125, 125, 125}
\definecolor{DarkBlue}{RGB}{50, 50, 153}
\definecolor{DarkGray}{RGB}{90,90,90}
\definecolor{LightGray}{RGB}{150,150,150}
\definecolor{TextGreen}{RGB}{115,155,15}
\definecolor{TextOrange}{RGB}{229, 148, 0}
\definecolor{Ocean}{RGB}{23,142,189}
\definecolor{BG}{RGB}{215,215,215}


\setbeamercolor{normal}{fg=DarkGray}
\setbeamercolor{title}{fg=UniBlue}
\setbeamercolor{frametitle}{fg=UniBlue}
\setbeamercolor{structure}{fg=UniBlue}
\setbeamercolor{normal text}{fg=DarkGray,bg=white}
\setbeamercolor{section number projected}{bg=UniBlue,fg=white}

\setbeamertemplate{itemize item}{\scriptsize\raise1.25pt\hbox{\donotcoloroutermaths$\blacktriangleright$}}
\setbeamertemplate{itemize subitem}{\tiny\raise1.5pt\hbox{\donotcoloroutermaths$\bullet$}}
\setbeamertemplate{itemize subsubitem}{\tiny\raise1.5pt\hbox{\donotcoloroutermaths$\blacksqaure$}}
\setbeamercolor{itemize item}{fg=darkred}
\setbeamercolor{itemize subitem}{fg=TextGreen}
\setbeamercolor{itemize subbody}{fg=LightGray}

\setbeamertemplate{enumerate subitem}{\insertenumlabel.\insertsubenumlabel}
\setbeamertemplate{enumerate subsubitem}{\insertenumlabel.\insertsubenumlabel.\insertsubsubenumlabel}
\setbeamertemplate{enumerate mini template}{\insertenumlabel}

\setbeamertemplate{navigation symbols}{}

\newcommand\VeryLargeFont{\fontsize{30}{15}\selectfont}
\newcommand\LargeFont{\fontsize{15}{15}\selectfont}
\newcommand\TinyFont{\fontsize{6}{6}\selectfont}

\setbeamertemplate{frametitle} {
  \nointerlineskip
  \begin{beamercolorbox}[sep=0.15cm,ht=1.3em,wd=\paperwidth]{frametitle}
    \vbox{}\vskip-2ex
    \strut\insertframetitle\strut
    \vskip-0.8ex
  \end{beamercolorbox}
}

\defbeamertemplate*{title page}{customized}[1][] {
	\centering
	\bigskip
	\bigskip
	\bigskip
	\usebeamercolor[fg]{title}\insertsubtitle\par
	\usebeamerfont{title}\inserttitle\par
	\usebeamerfont{subtitle}
	\bigskip
	\usebeamercolor[fg]{normal}
	\usebeamerfont{author}\insertauthor\par
	\usebeamerfont{institute}\insertinstitute\par
	\usebeamerfont{date}\insertdate\par
	\bigskip
	\bigskip
	\bigskip
	\bigskip
	\bigskip
	\usebeamercolor[fg]{titlegraphic}\inserttitlegraphic
}

%----------------------------------------------------------------------------------------
%	TITLE PAGE
%----------------------------------------------------------------------------------------
\title[Operating Systems of the IoT]{An Introduction to the Operating Systems of the IoT}
\author{Elahe Jalalpoor and Parham Alvani}
\institute[AUT] {
  Amirkabir University of Technology \\
  \medskip
  {\small\tt elahejalalpoor@gmail.com}\\
  {\small\tt parham.alvani@gmail.com}
  \medskip
}
\date{Jun 22, 2015}
\titlegraphic{\hspace*{5cm}\includegraphics[width=2cm]{figs/aut_logo.jpeg}}

\begin{document}

\begin{frame}
\titlepage
\end{frame}

%----------------------------------------------------------------------------------------
%	PRESENTATION SLIDES
%----------------------------------------------------------------------------------------

%------------------------------------------------
\begin{frame}
	\begin{columns}
		\begin{column}{10cm}
			\vspace{2cm}
			\begin{block}{
				\centering\textcolor{darkred}{What is IoT...}}
				\justifying
				The Internet of Things (IoT) is a scenario in which objects, animals or people are provided with unique identifiers and the ability to transfer data over a network without requiring human-to-human or human-to-computer interaction.\\
			\end{block}
		\end{column}
	\end{columns}
	\vspace{.75cm}
	\hspace*{8.5cm}\includegraphics[width=3cm]{figs/Internet-of-Things-1.jpg}
\end{frame}

%------------------------------------------------
\begin{frame}
	\frametitle{Open Source Operating Systems for the IoT}
	\begin{columns}[c]
		\begin{column}{30cm}
			\vspace{.1cm}
			\begin{itemize}
				\justifying
				\item \textcolor{blue}{\href{http://www.freertos.org}{FreeRTOS}}
				\item \textcolor{blue}{\href{http://www.riot-os.org}{RIOT}}
				\item \textcolor{blue}{\href{http://www.contiki-os.org}{Contiki}}
				\item TinyOS
				\item Embedded Linux
				\item \textcolor{blue}{\href{http://www.openwsn.org}{OpenWSN}}
			\end{itemize}
		\end{column}
	\end{columns}
	\vspace{.5cm}
	\hspace*{5.5cm} \includegraphics[width=5cm]{figs/Internet-of-Things-2.jpg}
\end{frame}

%------------------------------------------------
\begin{frame}
	\frametitle{FreeRTOS}
	\begin{columns}[c]
		\begin{column}{30cm}
			\vspace{.1cm}
			\begin{itemize}
				\justifying
				\item FreeRTOS is designed to be \textcolor{TextOrange}{small}
				and \textcolor{TextOrange}{simple}.
				\item The kernel itself consists of only three or four C files.
				\item It provides methods for multiple threads or tasks, mutexes,\\
				semaphores and software timers.
				\item Key features are \textcolor{TextGreen}{very small memory footprint},
				\textcolor{TextGreen}{low overhead},\\
				and \textcolor{TextGreen}{very fast execution}.
			\end{itemize}
		\end{column}
	\end{columns}
	\vspace{.5cm}
	\hspace*{5.5cm} \includegraphics[width=5cm]{figs/freertos-logo.jpg}	
\end{frame}

%------------------------------------------------
\begin{frame}
	\frametitle{RIOT}
	\begin{columns}[c]
		\begin{column}{30cm}
			\vspace{.1cm}
			\begin{itemize}
				\justifying
				\item RIOT is a \textcolor{TextOrange}{real-time}
				\textcolor{TextOrange}{multi-threading} operating system.
				\item RIOT is based on design objectives including:
				\begin{itemize}
					\justifying
					\item Energy-Efficiency
					\item Reliability
					\item Real-Time Capabilities
					\item Small Memory Footprint
					\item Modularity
					\item Uniform API Access\\
					independent of the underlying hardware\\
					(this API offers partial POSIX compliance)
				\end{itemize}
			\end{itemize}
		\end{column}
	\end{columns}
	\vspace{.5cm}
	\hspace*{5.5cm} \includegraphics[width=5cm]{figs/riot-logo.png}
\end{frame}

%------------------------------------------------
\begin{frame}
	\frametitle{Contiki}
	\begin{columns}[c]
		\begin{column}{30cm}
			\vspace{.1cm}
			\begin{itemize}
				\justifying
				\item Contiki is an open source operating system for \textcolor{TextOrange}{networked},\\
				\textcolor{TextOrange}{memory-constrained} systems
				\item Contiki provides three network mechanisms:
				\begin{itemize}
					\justifying
					\item The uIP stack, which provides IPv4 networking,
					\item The uIPv6 stack, which provides IPv6 networking,
					\item The Rime stack, which is a set of custom lightweight networking protocols\\
					designed specifically for low-power wireless networks.
				\end{itemize}
			\end{itemize}
		\end{column}
	\end{columns}
	\vspace{.5cm}
	\hspace*{5.5cm} \includegraphics[width=5cm]{figs/contiki-logo.png}
\end{frame}

%------------------------------------------------
\begin{frame}
	\frametitle{TinyOS}
	\begin{columns}[c]
		\begin{column}{30cm}
			\vspace{.1cm}
			\begin{itemize}
				\justifying
				\item TinyOS is a \textcolor{TextOrange}{component-based} operating system and platform\\
				targeting wireless sensor networks.
				\item TinyOS is an embedded operating system written in the \textcolor{TextOrange}{nesC}\\
				\textcolor{TextOrange}{programming language} as a set of cooperating tasks and processes.
			\end{itemize}
		\end{column}
	\end{columns}
	\vspace{.5cm}
	\hspace*{5.5cm} \includegraphics[width=5cm]{figs/tinyos-logo.jpg}
\end{frame}

%------------------------------------------------
\begin{frame}
	\frametitle{Embedded Linux}
	\begin{columns}[c]
		\begin{column}{30cm}
			\vspace{.1cm}
			\begin{itemize}
				\justifying
				\item Embedded Linux is created using OpenEmbedded,\\
				the build framework for embedded Linux.
				\item OpenEmbedded offers a best-in-class cross-compile environment.
			\end{itemize}
		\end{column}
	\end{columns}
	\vspace{.5cm}
	\hspace*{5.5cm} \includegraphics[width=5cm]{figs/linux-logo.jpeg}
\end{frame}

%------------------------------------------------
\begin{frame}
	\frametitle{OpenWSN}
	\begin{columns}[c]
		\begin{column}{30cm}
			\vspace{.1cm}
			\begin{itemize}
				\justifying
				\item The goal of the OpenWSN project is to provide open-source\\
				implementations of a complete protocol stack based on Internet of\\
				Things standards, on a variety of software and hardware platforms.
			\end{itemize}
		\end{column}
	\end{columns}
	\vspace{.5cm}
	\hspace*{5.5cm} \includegraphics[width=5cm]{figs/openwsn-logo.png}
\end{frame}

%------------------------------------------------
\begin{frame}
	\frametitle{Comparison}
	\begin{columns}[c]
		\begin{column}{30cm}
			\hspace{0.9cm}
			\begin{tabular}{ | c | c | c | c | c |}
				\hline
				OS & Min RAW & Min ROM & C Support & C++ Support \\ \hline
				Contiki & $< 2kB$ & $< 30kB$ & \textcolor{TextOrange}{Partial support} &
				\textcolor{red}{No support} \\ \hline
				Tiny OS & $< 1kB$ & $< 4kB$ & \textcolor{red}{No support} &
				\textcolor{red}{No support} \\ \hline
				Linux & $\sim 1MB$ & $\sim 1MB$ & \textcolor{TextGreen}{Full support} &
				\textcolor{TextGreen}{Full support} \\ \hline
				RIOT & $\sim 1.5kB$ & $\sim 5kB$ & \textcolor{TextGreen}{Full support} &
				\textcolor{TextGreen}{Full support} \\ \hline
			\end{tabular}
		\end{column}
	\end{columns}
	\vspace{.5cm}
	\hspace*{1cm}
	\includegraphics[width=2cm]{figs/contiki-logo.png}
	\hspace*{.5cm}
	\includegraphics[width=2cm]{figs/tinyos-logo.jpg}
	\hspace*{.5cm}
	\includegraphics[width=2cm]{figs/linux-logo.jpeg}
	\hspace*{.5cm}
	\includegraphics[width=2cm]{figs/riot-logo.png}
\end{frame}

%------------------------------------------------
\begin{frame}
	\frametitle{Comparison}
	\begin{columns}[c]
		\begin{column}{30cm}
			\hspace{0.9cm}
			\begin{tabular}{ | c | c | c | c |}
				\hline
				OS & Multi-Threading & Modularity & Real-Time \\ \hline
				Contiki & \textcolor{TextOrange}{Partial support} &
				\textcolor{TextOrange}{Partial support} &
				\textcolor{TextOrange}{Partial support} \\ \hline
				Tiny OS & \textcolor{TextOrange}{Partial support} &
				\textcolor{red}{No support} &
				\textcolor{red}{No support} \\ \hline
				Linux & \textcolor{TextGreen}{Full support} &
				\textcolor{TextOrange}{Partial support} &
				\textcolor{TextOrange}{Partial support} \\ \hline
				RIOT & \textcolor{TextGreen}{Full support} &
				\textcolor{TextGreen}{Full support} &
				\textcolor{TextGreen}{Full support} \\ \hline
			\end{tabular}
		\end{column}
	\end{columns}
	\vspace{.5cm}
	\hspace*{1cm}
	\includegraphics[width=2cm]{figs/contiki-logo.png}
	\hspace*{.5cm}
	\includegraphics[width=2cm]{figs/tinyos-logo.jpg}
	\hspace*{.5cm}
	\includegraphics[width=2cm]{figs/linux-logo.jpeg}
	\hspace*{.5cm}
	\includegraphics[width=2cm]{figs/riot-logo.png}
\end{frame}

%------------------------------------------------
\begin{frame}
	\frametitle{Why Not Linux?}
	\begin{columns}[c]
		\begin{column}{12cm}
			\vspace{1cm}
			\begin{block}{
				\centering\textcolor{darkred}{Real-Time Linux}}
				\justifying
				Controlling a laser with Linux is crazy, but everyone in this room is crazy
				in his own way. So if you want to use Linux to control an industrial welding
				laser, I have no problem with your using PREEMPT\_RT.
				\vspace{.2cm}
				\hspace*{9.5cm}\footnotesize{- Linus Torvalds}
			\end{block}
		\end{column}
	\end{columns}
	\vspace{.5cm}
	\hspace*{.5cm}
	\includegraphics[width=2cm]{figs/preempt-rt.png}
	\hspace*{3cm}
	\includegraphics[width=5cm]{figs/linus-torvalds.jpg}
\end{frame}

%------------------------------------------------
\begin{frame}
	\frametitle{Why Not Linux?}
	\begin{columns}[c]
		\begin{column}{30cm}
			\vspace{.1cm}
			\begin{itemize}
				\justifying
				\item Linux certainly is a robust, developer-friendly OS
				\item Linux has a disadvantage when compared to a real-time operating system:
				\begin{itemize}
					\justifying
					\item Memory footprint
					\item It simply will not run on 8 or 16-bit MCUs
				\end{itemize}
			\end{itemize}
		\end{column}
	\end{columns}
	\vspace{.5cm}
	\hspace*{5.5cm} \includegraphics[width=5cm]{figs/tux-sad.png}
\end{frame}

%------------------------------------------------
\begin{frame}
	\frametitle{Close Source Operating Systems for the IoT}
	\begin{columns}[c]
		\begin{column}{30cm}
			\vspace{.1cm}
			\begin{itemize}
				\justifying
				\item \textcolor{blue}{\href{https://mbed.org/}{ARM mbed}}
				\item Huawei LiteOS
				\item Google Brillo
			\end{itemize}
		\end{column}
	\end{columns}
	\vspace{.5cm}
	\hspace*{5.5cm} \includegraphics[width=5cm]{figs/Internet-of-Things-2.jpg}
\end{frame}

%------------------------------------------------
\begin{frame}
	\frametitle{ARM mbed}
	\begin{columns}[c]
		\begin{column}{30cm}
			\vspace{.1cm}
			\begin{itemize}
				\justifying
				\item Automation of power management
				\item Software asset protection and secure firmware updates for\\
				device security \& management
				\item Connectivity protocol stack support for Bluetooth® low energy,\\
				Cellular, Ethernet, Thread, Wi-fi®,  Zigbee IP, Zigbee NAN, 6LoWPAN
			\end{itemize}
		\end{column}
	\end{columns}
	\vspace{.5cm}
	\hspace*{5.5cm} \includegraphics[width=5cm]{figs/ARM-mbed-logo.png}
\end{frame}

%------------------------------------------------
\begin{frame}
	\frametitle{Huawei LiteOS}
	\begin{columns}[c]
		\begin{column}{30cm}
			\vspace{.1cm}
			\begin{itemize}
				\justifying
				\item The company says that its \textcolor{TextOrange}{LiteOS} is
				the \textcolor{TextGreen}{lightest} software of\\
				its kind and can be used to power a range of smart devices
			\end{itemize}
		\end{column}
	\end{columns}
	\vspace{.5cm}
	\hspace*{5.5cm} \includegraphics[width=5cm]{figs/huawei-liteos-logo.jpg}
\end{frame}

%------------------------------------------------
\begin{frame}
	\frametitle{Google Brillo}
	\begin{columns}[c]
		\begin{column}{30cm}
			\vspace{.1cm}
			\begin{itemize}
				\justifying
				\item Brillo is \textcolor{TextGreen}{derived} from Android
				but \textcolor{TextGreen}{polished} to just the lower levels.
				\item It supports Wi-Fi, Bluetooth Low Energy, and other Android things.
			\end{itemize}
		\end{column}
	\end{columns}
	\vspace{.5cm}
	\hspace*{5.5cm} \includegraphics[width=5cm]{figs/google-brillo-logo.jpeg}
\end{frame}

%------------------------------------------------
\begin{frame}
	\frametitle{Requirements for IoT}
	\begin{columns}[c]
		\begin{column}{30cm}
			\vspace{.1cm}
			\begin{itemize}
				\justifying
				\item Scalability
				\item Modularity
				\item Connectivity
				\item Reliability
			\end{itemize}
		\end{column}
	\end{columns}
	\vspace{.5cm}
	\hspace*{5.5cm} \includegraphics[width=5cm]{figs/Internet-of-Things-3.jpg}
\end{frame}

%------------------------------------------------
\begin{frame}
	\frametitle{Internet Usage and Protocols for the IoT}
	\begin{columns}[c]
		\begin{column}{30cm}
			\vspace{.1cm}
			\begin{itemize}
				\justifying
				\item Can you build an IoT system with familiar Web technologies?
			\end{itemize}
		\end{column}
	\end{columns}
	\vspace{.5cm}
	\hspace*{5.5cm}
\end{frame}

%------------------------------------------------
\begin{frame}
	\frametitle{Internet Usage and Protocols for the IoT}
	\begin{columns}[c]
		\begin{column}{30cm}
			\vspace{.1cm}
			\begin{itemize}
				\justifying
				\item Can you build an IoT system with familiar Web technologies?
				\item Yes you can, although the result would not be as \textcolor{Ocean}{efficient} as\\
				with the \textcolor{TextGreen}{newer protocols}.
			\end{itemize}
		\end{column}
	\end{columns}
	\vspace{.5cm}
	\hspace*{5.5cm}
\end{frame}

%------------------------------------------------
\begin{frame}
	\frametitle{Internet Usage and Protocols for the IoT}
	\vspace{.5cm}
	\hspace*{1.5cm} \includegraphics[width=10cm]{figs/Web-and-IoT-Stacks.png}
\end{frame}

%------------------------------------------------
\begin{frame}
	\frametitle{Internet Usage and Protocols for the IoT}
	\vspace{.25cm}
	\hspace*{.75cm} \includegraphics[width=10.5cm]{figs/iot-network-protocols.jpeg}
\end{frame}

%------------------------------------------------
\begin{frame}
	\frametitle{Modularity}
	\begin{columns}[c]
		\begin{column}{30cm}
			\vspace{.1cm}
			\begin{itemize}
				\justifying
				\item IoT device will require a modular operating system that separates\\
				\textcolor{TextGreen}{the core kernel} from
				\textcolor{TextOrange}{middleware},
				\textcolor{TextOrange}{protocols},
				and \textcolor{TextOrange}{applications}.
			\end{itemize}
		\end{column}
	\end{columns}
	\vspace{.5cm}
	\hspace*{5.5cm}
\end{frame}

%------------------------------------------------
\begin{frame}
	\frametitle{Multi-Tasking, Thread Model}
	\begin{columns}[c]
		\begin{column}{30cm}
			\vspace{.1cm}
			\begin{itemize}
				\justifying
				\item Most RTOS products on the market are thread model.
				\item \textcolor{TextGreen}{Tasks} are now called \textcolor{TextGreen}{threads}.
				\item All the \textcolor{TextOrange}{tasks} code and data occupy
				\textcolor{Ocean}{the same address space},\\
				along with that of the RTOS itself.
			\end{itemize}
		\end{column}
	\end{columns}
	\vspace{.5cm}
	\hspace*{5.5cm} \includegraphics[width=5cm]{figs/thread-model.jpg}
\end{frame}

%------------------------------------------------
\begin{frame}
	\vspace{1cm}
	\begin{Huge}
		\begin{center}
			\usebeamercolor[fg]{title}Questions?
		\end{center}
	\end{Huge}
\end{frame}

\end{document} 
